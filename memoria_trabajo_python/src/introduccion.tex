\section{Introducción}

En este trabajo resolveremos un problema de clasificación utilizando el lenguaje de programación Python. En concreto utilizaremos las distintas técnicas de preprocesamiento, selección de hiperparámetros y aprendizaje automático disponibles en el paquete scikit-learn, así como otras bibliotecas para diversas tareas entre las que se encuentran visualizar gráficamente el conjunto de datos, guardar los modelos obtenidos en disco, entre otras tareas.

En concreto, los módulos utilizados son:

\begin{itemize}
	\item \texttt{os}: Realizar tareas del sistema operativo, como crear carpetas.
	\item \texttt{sys}: Gestionar parámetros específicos del sistema.
	\item \texttt{pandas}: Leer ficheros de disco en forma de DataFrames.
	\item \texttt{seaborn}: Herramienta para realizar gráficas.
	\item \texttt{matplotlib}: Herramienta para realizar gráficas.
	\item \texttt{numpy}: Herramientas de computación numérica.
	\item \texttt{pickle}: Almacenar variables de Python en disco.
	\item \texttt{sklearn}: Herramientas de aprendizaje automático.
\end{itemize}

\newpage
